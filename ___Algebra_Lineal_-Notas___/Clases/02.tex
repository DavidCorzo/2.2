\section{2.2 Métodos directos para resolver sistemas de ecuaciones}
Una matriz es un arreglo rectngular con $m$ filas y $n$ columnas.
\begin{center}
   \begin{align*}
       A = \begin{bmatrix}
            1&4&8 \\ 
            3&0&2 \\ 
            7&0&1 \\ 
       \end{bmatrix} 
        \qq \qq 
        B = \begin{bmatrix}
            1&2 \\
            0&1 \\
            3&4 \\
        \end{bmatrix} \\ 
   \end{align*}
\end{center}
Se denota el tamaño como: $m\times n$  filas $\times $ columnas.
\begin{center}
   \begin{align*}
       A: 3\times 3 \qq \qq 
       B: 3\times 2 \qq \qq 
       C: 1\times 4 \qq \qq  \\ 
   \end{align*}
\end{center}
Dado el sistema de ecuaciones lineales.
\begin{center}
   \begin{align*}
       x_1+2x_2 3x_3+4x_4=b_1 \\ 
       x_1+3x_2 4x_3+5x_4=b_1 \\ 
       x_1+4x_2 5x_3+6x_4=b_1 \\ 
   \end{align*}
\end{center}
Si los coheficientes se guardan en una matriz $A$. Y los términos constantes en un vector comuna $b$, entonces el sistema se puede representar con una matriz aumentada $[A|B]$.
\begin{center}
   \begin{align*}
       \begin{bmatrix}
           1&2&3&4 &b_1 \\ 
           1&3&4&5 &b_2 \\ 
           1&4&5&6 &b_3 \\ 
           x_1&x_2&x_3&x_4&\\ 
       \end{bmatrix}
   \end{align*}
\end{center}

\section{Ejercicios}
Escriba la matriz aumentada del sistema dado. 
\begin{enumerate}
    \item \begin{center}
       \begin{align*}
           \begin{array}{cc}
                x+2y+3z&=6 \\ 
                2x+y+4z&=8 \\ 
                x+y+z&=4 \\ 
           \end{array} 
           \qq 
           \begin{bmatrix}
                1 & 2 & 3 & 6 \\ 
                2 & 1 & 4 & 8 \\ 
                1 & 1 & 1 & 4 \\ 
           \end{bmatrix}
           \to 
           \begin{bmatrix}
               1&&&c_1 \\ 
               &1&&c_2 \\ 
               &&1&c_3 \\ 
           \end{bmatrix}
       \end{align*}
    \end{center}

    
    \item \begin{center}
       \begin{align*}
            \begin{array}{cc}
                2x_1+x_2&=8 \\ 
                4x_1+x_2&=6 \\ 
                6x_1+x_2&=2 \\ 
            \end{array}
            \qq \begin{bmatrix}
                2&-1&8 \\
                4&-1&6 \\
                6&-1&2 \\
            \end{bmatrix}
       \end{align*}
    \end{center}
\end{enumerate}
Matrices en forma escalonada por renglones. Abreviad como FER: 
\begin{enumerate}
    \item Cualquier fila que tenga solo ceros se ubica en la parte interior de la matriz.
    \item La entrada principal de cada fila es la entrada que está más a la izquierda en la fila. 
    \item En cada renglón diferente de cero, todas las entradas debajo y la izquierda de la entrada principal son ceros.
        \begin{center}
           \begin{align*}
               \begin{bmatrix}
                   1&2&3&6 \\ 
                   0&0&1&2 \\ 
                   0&0&0&5 \\ 
               \end{bmatrix}
               \qq \qq 
               \begin{bmatrix}
                   1&2&3 \\ 
                   0&1&1 \\ 
                   0&0&1 \\ 
               \end{bmatrix}
               \qq \qq 
               \begin{bmatrix}
                    0&1&1&1 \\ 
                    0&0&0&2 \\ 
                    0&0&0&0 \\ 
                    0&0&0&0 \\ 
               \end{bmatrix}
           \end{align*}
        \end{center}
\end{enumerate}

\section{Ejercicio 2}Determine si la matriz está en FER. 
\begin{center}
   \begin{align*}
       A = \begin{bmatrix}
           1&4&2&1 \\ 
           1&0&0&0 \\ 
           0&0&1&0 \\ 
       \end{bmatrix} \qq \qq \text{ No, hay 1 debajo de la entrada principal $a_{11}$} \\ 
       \\ 
       B = \begin{bmatrix}
           1&1&0&2 \\ 
           0&0&1&2 \\ 
           0&0&0&4 \\ 
       \end{bmatrix} \qq \qq \text{ $B$ Sí está en FER. } \\ 
       \\ 
       C = \begin{bmatrix}
           0&0&2 \\ 
           0&2&1 \\ 
           3&2&1 \\ 
       \end{bmatrix} \qq \qq \text{ C no está en FER }. \\ 
   \end{align*}
\end{center}
Pero si se intercambian sus 1era y 3era filas si lo es. $R_1 \longleftrightarrow R_3$.
\begin{center}
   \begin{align*}
       C_1 = \begin{bmatrix}
           3&2&1 \\ 
           0&2&1 \\ 
           0&0&2 \\ 
       \end{bmatrix} \qq \qq \text{ Si está en FER. } \\ 
   \end{align*}
\end{center}
Si se realizan operacines elementales de renglón sobre una matriz se puede una nueva matriz que sigue manteniendo la misma solución. ``Sistema equiv''.
\subsection{Operaciones elementales de renglón}
\begin{enumerate}
    \item Intercambio de renglones $R_i \longleftrightarrow R_j$ 
    \item Multiplicación por una constante $kR_i \qq k\neq 0$ 
    \item 
\end{enumerate}
