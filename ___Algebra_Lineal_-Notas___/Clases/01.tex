\section{Sistemas de ecuaciones lineales}
\begin{itemize}
    \item Ec. recta en $\mathbb{R}^2$  $ax+by=c$.
    \item Ec. recta en $\mathbb{R}^3$  $ax+by+cz=d$.
    \item Ex. Lineal en n variables: $a_1x_1+a_2x_2+\dots+a_nx_n=b$ 
    \item Coheficientes $a_1,a_2,\dots,a_n$ $a_i \in \mathbb{R}$ 
    \item Término constante: b
\end{itemize}

\subsection{Ejemplos}
\begin{center}
   \begin{align*}
        3x_1-9x_2+12 = 0 \\ 
        \sqrt{2}w+\ln\p{ 8 } x+e^{10}y+\frac{1}{\pi}z  = 5  \\ 
   \end{align*}
\end{center}
\begin{itemize}
    \item Son ecuaciones lineales. 
\end{itemize}

\subsection{Ecuaciones que no son lineales}
\begin{itemize}
    \item Producto entre variables $xy+yz=8$ NO ES
    \item Potencia diferente de uno $x^2+\sqrt{y}+z^{-1}=8$ NO ES
    \item Funciones no lineales: $\sqrt{2w}+\ln\p{ 8x } +e^{10y} = \sin\p{ 3 } $ 
\end{itemize}

\subsection{Solución de una ecuación lineal}
\begin{itemize}
    \item La solución es un vector $[S_1,S_2,...S_n]$  cuyos componentes si satisfacen la ec. $a_1x_1+a_2x_2+\dots+a_nx_=b$  cuando se substituyen las $S_i$.
\end{itemize}
Ejemplo: considere la ec. lineal $3x_1-9x_2+12= 0$
\begin{center}
   \begin{align*}
       \text{ Resuelva para $x_i$:  } \begin{array}{l}
           x_1 = 3t-4 \\ 
           x_2= t \\ 
       \end{array} \qq \qq \vec{S} = \begin{bmatrix}
            3t-4 \\ 
            t \\ 
       \end{bmatrix}\\ 
   \end{align*}
   \begin{itemize}
       \item $t$, parámetro es cualquier número real.
       \item Verifique: $9t-12-9t+12=0$ \checkmark  
   \end{itemize}
\end{center}

Sistema de ecuaciones lineales: es un conjunto finito de ecuaciones lineales. $m$ ecuaciones de $n$ variables.

\begin{center}
   \begin{align*}
       \begin{array}{l}
           x + y+z=8 \\ 
           x+y+z=4 \\ 
           2x+z = 4 \\
       \end{array} 
       \qq \qq \begin{array}{l}
           x+y=8 \\ 
           x+y=10 \\ 
           x+3y=5 \\ 
       \end{array} \\ 
       \\ 
       x+y+z=4 \qq \qq \text{$m $ y $n$ } \\ %%%%%
   \end{align*}
\end{center}

Solución de un sistema de ecuaciones: es un vector $\vec{S}$ que satisface todas las ecs. del sistema.

\subsection{Ejercicio 1}
Resuelva los siguientes sistemas (pg12).
\begin{enumerate}
    \item 
        \begin{center}
            \begin{align*}
                \begin{array}{l}
                    x+y=2 \\ 3x+3y=6 \\ 
                \end{array}
                \qq \qq 
                \text{ 2 ecs $\times$ 2 variables } \\ 
            \end{align*}
        \end{center}
        \begin{itemize}
            \item Métodos: Algebraico (eliminación, sustitución, igualación).
            \item Gráfico: la sola intersección entre las rectas.
        \end{itemize}
        \begin{center}
           \begin{align*}
               R_2-3R_1: \qq 0+0=0 \qquad \implies 0=0 \qquad \text{ tautología. } \\ 
               \frac{1}{3} R_2: \qq x+y=2 \qquad \text{ la recta está repetida. } \\ 
               \text{ La solución es $y=2-x$ } \\ 
               \begin{array}{l}
                   x=t \\ 
                   y=2-t \\ 
               \end{array} \qquad 
               \begin{array}{l}
                   \vec{S} = \begin{bmatrix}
                       t \\ 2-t \\ 
                   \end{bmatrix}
               \end{array}
               \begin{array}{lll}
                    \begin{bmatrix}
                        0 \\ 2 \\ 
                    \end{bmatrix} &
                    \begin{bmatrix}
                        1 \\1  \\ 
                    \end{bmatrix} &
                    \begin{bmatrix}
                        2 \\ 0 \\ 
                    \end{bmatrix} &
                    ...
                   \\ 
               \end{array} 
            \\ 
               \text{ Sustitución:  } \begin{array}{l}
                   y=2-x \\ 
                   3x+6-3x=6 \\ 
               \end{array}
               \qq \implies  6 = 6 \\ 
               
           \end{align*}
        \end{center}
\end{enumerate}
