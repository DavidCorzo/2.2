\section{...}


\section{Qué indican las cuentas}
\begin{itemize}
    \item Activos
        \begin{itemize}
            \item Se enlistan en base a su liquidez.
            \item De los más líquidos a los ilíquidos.
            \item Los más líquidos siendo cuentas como inventario, bancos e inversiones.
        \end{itemize}
    
    \item Pasivos 
        \begin{itemize}
            \item Se enlistan en función de cuál debe de pagarse primero.
            \item Las primeras cuentas podrían ser cuentas por pagar, anticipos de clientes, 
        \end{itemize}
    
    \item Capital 
        \begin{itemize}
            \item Acciones preferentes: la diferencia entre una acción preferente y una acción común son que son a corto plazo y no te dan derecho a votar dentro de la empresa. 
            \item Acciones comunes: Las acciones comunes tiene derecho a voz pero no a voto dentro de una empresa.
            \item Utilidades retenidas: es la proporción de utilidades en el año que se deciden retener para reinvertirlo en la empresa. Este no es dinero que se da en dividendos, ya se decidió que se reinvertirá de nuevo en la empresa. Si me salgo de la empresa no tengo derecho a esas utilidades retenidas. 
        \end{itemize}
\end{itemize}

\section{Balance general}
\begin{itemize}
    \item Activo circulante: efectivo, cuentas por cobrar, inventario. 
    \item Activos fijos: propiedad, planta y equipo. 
    \item Pasivos circulantes: cuentas por pagar, pasivo acumulado, documentos por pagar. 
    \item Pasivos (LP): deudas a LP, bonos (los bonos tienen un interés fijos, a diferencia de las deudas). 
\end{itemize}

\section{Qué significa?}
\[
  \text{ Capital de trabajo } = \text{ Activo }
\]
\[
  \text{...} 
\]

\section{Qué indican las cuentas?}
\begin{itemize}
    \item Inventario: PEPS (FIFO), UEPS (LIFO).
    \item Depreciación / amortización: acelerada - fines fiscales, Línea recta. (Las empresas pueden manejar diferentes estados financieros).
    \item Capital contable: acción común al valor par, aportaciones a capital, utilidades retenidas.
    \item ... 
\end{itemize}

\section{Estado de resultados}
\begin{itemize}
    \item Representa las operaciones durante un periodo específico, usualmente un trimestre o un año.
    \item El rubro más importante de este es las utilidades o utiidades por acción. 
    \item Tipo de financiamiento:
        \begin{itemize}
            \item Acciones: comunes, preferentes. 
            \item Deuda: banco, bonos.
        \end{itemize}
    
    \item El estado de resultados se genera en base al método contable de lo devengado. Es decir, los ingresos \textbf{se reconocen cuando se ganaron}, no cuando se recibe el efectivo, y los \textbf{gastos se reconocen cuando se incurren}, no cuando se pagan. En consecuencia no todos los montos del estado de resultados representa flujos de efectivos.
\end{itemize}

\section{Flujos de efectivo}
\begin{itemize}
    \item El valor de un activo se determina por el flujo de efectivo que el mismo genera. 
    \item La utilidad neta es importante, pero los flujos de efectivo ...
\end{itemize}

...


