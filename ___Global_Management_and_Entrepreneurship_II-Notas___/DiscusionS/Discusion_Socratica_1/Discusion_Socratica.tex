\documentclass{article}

\usepackage{generalsnips}
\usepackage{calculussnips}
\usepackage[margin = 1in]{geometry}
\usepackage[spanish]{babel}
\usepackage{amsmath}
\usepackage{amsthm}
\usepackage[utf8]{inputenc}
\usepackage{hyperref}
\title{Discusión Socrática \# 1 - Caso Huawei}
\date{2020 August 11} % , 05:35PM
\author{David Corzo }
\begin{document}
\maketitle
%%%%%%%%%%%%%%%%%%%%%%%%%%%%%%%%%%%%%%%%%%%%%%%%%%%%%%%%%%%%%%%%%%%%%%%%%%%%%%%%%%%%%%%%%%%%%%%%%%%%%%%%%%%%%%%%%%%%%%%%%%%%%%%%%%%%%%%

\section{Síntesis de la lectura}
% 1. Síntesis de la lectura. Debe expresarse con sus palabras y con la estructura que considere apropiada a fin de entender en la síntesis las ideas centrales del documento en cuestión. (250 palabras) 
En síntesis, Huawei es lo que es hoy gracias a su inteligente y acertada forma de usar la innovación. El hecho que tienen un departamento muy bien financiado dedicado a la innovación llamado HIRP ó Huawei Innovation Research Program demuestra esto. HIRP se centra en trabajar con universidades para hacer nuevas innovaciones que pueden resultar en mejores productos Huawei. HIRP ha entrado en colaboraciones con más de 300 universidades en 20 diferentes países, recaudan a personas altamente técnicas y especializadas para participar en las colaboraciones y para muchas es una primera oportunidad de aplicar conceptos teóricos y hacerlos una realidad. HIRP ha desarrollado una forma sistemática para la gestión de proyectos, cada equipo encargado de un proyecto tiene su propio \emph{Project Management Team} para asegurar que las cosas salgan a tiempo y con calidad. Basado en el plan estratégico, Huawei define \emph{Research Targets} u objetivos de investigación, en cada departamento se le asigna alguno de estos objetivos a ingenieros individuales, los ingenieros pueden decidir formar grupos para lograr estos objetivos y colaborarse entre sí, de ser necesario buscar a más personas con quienes colaborar se buscan ingenieros recomendados de empresas de confianza. Los resultados de estos objetivos son los que posteriormente son publicados como \emph{Academic papers} e incorporados al producto final. Este proceso permite que Huawei pueda hacer grandes innovaciones y poder crecer tanto. Este constante flujo de innovación es el que ha permitido a Huawei ser el dueño de la mayor cantidad de patentes en China. Todo esto contribuye a Huawei ser una empresa que ya crecido mucho, el enfoque en la innovación y la manera en la que la aplican es representa la forma eficiente y eficaz de lograr avances tecnológicos que terminan beneficiando a los usuarios y a Huawei.

\section{Referencias bibliográficas}
% 2. Referencias bibliográficas: Investigar dos referencias e indicar cómo éstas se relacionan con el argumento principal de la lectura. (40 palabras por referencia aproximadamente) 
\subsection{``Open innovation networks between academia and industry: an imperative for breakthrough therapies''}
En este artículo se puede observar claramente la relación entre lo académico y lo industrial como una relación de mutuo beneficio en muchas cosas. Como producto natural de esta relación se obtiene la innovación, pero además de innovación se producen efectos como los de la efectividad y eficacia. Esto impacta en los costos. Los costos es algo que ninguna compañía puede descuidar, la innovación vuelve procesos lentos y retrógrados en procesos costo-efectivos y rápidos, y todo esto se debe a la alianza de la industria y la academia. Huawei tiene una de estas alianzas entre la industria y la academia y ya ha observado los frutos de esta relación en todo desde sus costos hasta su competitividad en el mercado. \cite{innovacad}

\subsection{``The Most Innovative Companies 2014: Breaking through is hard to do''}
En este reporte elaborado por \emph{The boston consulting group} se presenta un gran conjunto de empresas basados en la innovación que tuvieron en el año 2014, algunas de las empresas más grandes se encuentran en esta lista desde BMW en 18, y algunos \emph{tech-giants} como Apple, Google, Samsung y Microsoft como primeros lugares. Huawei está en el lugar número 50 de la lista de empresas innovadoras lo cual se relaciona con el argumento de la lectura en el hecho que se está analizando el éxito de empresas innovadoras. \cite{innovative}



\section{Referencia que no se encuentra en el texto}
% 3. Investigar por lo menos una referencia (autor) que no se encuentre en las referencias de la lectura pero que tenga relación con el argumento principal del texto. (40 palabras aproximadamente) 
Una de las formas más efectivas de mantener una empresa competitiva es por medio de la innovación, todos los empresarios quieren crear \emph{game-changing-products} \cite{arti}, relacionándolo con el argumento de la lectura, propone la innovación como una forma de mantenerse altamente competitivo, sin embargo resalta lo difícil que es que una empresa lo considere, innovar no es algo que todos entienden y además muchas veces sale caro. Si no está siendo bien financiado el departamento de innovación, que es el caso de muchas empresas, pueden resultar en proyectos de investigación fallidos. Sin embargo empresas que tienen departamentos de innovación como HIRP de Huawei, Coca-Cola, Motorola y 3M ven un gran beneficio derivado. Muchas empresas líder como estas tienen un departamento de innovación tan sofisticado como el de Huawei, Coca-cola acredita su éxito en incremento de ventas a ``marketing e innovación'' \cite{inov}.

\section{Dos conceptos o términos desconocidos}
% 4. Identificar por lo menos dos conceptos o términos desconocidos por el estudiante que se encuentren en la lectura, y explicar en aproximadamente 30 palabras cada término. 
\textbf{Project management:} es la aplicación metodológica de conocimiento, habilidades, soft-skills y técnicas a un proyecto y sus actividades para poder cumplir con los requisitos del proyecto. \cite{pm} \par 
\textbf{CEO by rotation:} una empresa que tiene varios CEOs y delega uno por cada época del año, de esta forma las empresas pueden mantenerse alertas y competitivas, ya que este sistema está diseñado por que diferentes épocas del año requieren diferentes destrezas. \cite{CEObr}

\section{Tres preguntas al autor del texto}
% 5. Indicar tres preguntas que el estudiante le haría al autor del texto. (60 palabras aproximadamente)
\begin{enumerate}
    \item \pregunta{Cuál es el rol del gobierno Chino siendo tan intrusivo en su economía a este proceso de innovación} 
    \item \pregunta{Qué problemas ha enfrentado utilizando este modelo de innovación constante aparte de fracasar en algunos \emph{research targets}} 
    \item \pregunta{En promedio cuánto financiamiento necesita un proyecto de investigación}  
\end{enumerate} 

\section{Reflexión del caso Huawei}
% 6. Reflexionando sobre el caso de Huawei: (Para éste inciso máximo 500 palabras) 


\subsection{¿Cómo se integra la innovación en Huawei? ¿Qué tipos de programas de innovación existían en la empresa?}
La innovación se integra mediante un programa llamado HIRP, este se encarga de colaborar con las más de 300 universidades para coordinar y realizar \emph{research targets}. El programa recluta a las personas más técnicas y especializadas (tales como PhD o profesores) para elaborar los proyectos, a cada proyecto se le es asignado un Project Management team para asegurar que todo vaya en orden. Antiguamente, antes de establecer el HIRP, se tenía la HSTF ó Huawei Science and Technology Fund encargada de la innovación.

\subsection{¿Cuáles son las ventajas y desventajas del HIRP? ¿Cómo HIRP puede ayudar a Huawei a ser más eficaz como empresa? ¿Qué cambios debería de llevar a cabo Huawei para ser más exitoso en términos de innovación?  }
Las ventajas de este programa eran que les permitía a muchos profesores universitarios financiar proyectos de investigación que no podrían financiar ellos mismos. Lograr aplicar teorías en la realidad, muchas veces por primera vez, y a muchos les permite salir del \emph{ivory tower} para considerar las aplicaciones en el mundo real que tienen sus teorías. Además los profesores podrían trabajar con el mejor equipo que no era posible tener en los laboratorios de universidades. \par
Algunas desventajas eran que no a todos los profesores les agradaba la idea, ya que Huawei es una empresa ejercía presión en terminar y realizar los proyectos de investigación, cosa que a algunos universitarios no les gustó para nada, esta presión en la opinión de muchos perjudica el proceso de innovar. La comunicación también era un problema, muchos profesores tenían formas distintas de trabajar a las de Huawei y les costaba entender qué quería Huawei además de estimar cuanto tiempo y esfuerzo les tardaría realizarlo. También encontrar gente competente en realizar algunos proyectos era difícil, se invierte mucho tiempo en encontrarlos, a veces los proyectos se descartan por no encontrar gente que esté capacitada para realizarlos. \par
Algunos cambios podrían ser tomar proyectos más grandes, tomar más riesgos, ya que la gran mayoría de proyectos resultan en éxito. 

\subsection{¿Qué retos encontró Huawei en cuanto a innovación? ¿Qué medidas tomaron para sobrepasarlos? }
Para poder resolver el problema de la comunicación Huawei hizo el HIRP en 2010 y en 2014 lanzó HIRP's \emph{Open Innovation Platform} cuya función era mejorar la comunicación. \par 
Otro problema era el \emph{Not-Invented-Here}, para solucionar esto Huawei organizaba conferencias periódicas y premiaba los proyectos, además empezaron a motivar a los integrantes de los equipos compartir sus experiencias. \par 
Adicionalmente, para resolver el problema de discreción se instituyeron \emph{non-disclosure agreements} o acuerdos de no divulgación y el uso de patentes para proteger la propiedad intelectual. 

\subsection{¿Cómo se encuentran conformados los equipos encargados de proyectos de innovación en Huawei? ¿Cómo cambia la integración de los equipos según los proyectos de innovación? ¿Cuál es la importancia de tener equipos multidisciplinarios en este tipo de proyectos? }

Los equipos son conformados por el personal que trabajaba directamente con Huawei tales como Project Managers, el equipo de management, los supervisores, etcétera y los que se reclutaban de las universidades tales como profesores, estudiantes de doctorados, universitarios, etcétera.

Los equipos cambian según la complejidad e importancia del proyecto. 

Tener equipos multidisciplinarios es de suma importancia, ya que asegura tener riqueza de puntos de vista para poder tomar una decisión consensuada acerca de cuál es el mejor plan de acción en ciertos escenarios, esto produce resultados efectivos, eficaces y de alta calidad.

\subsection{¿Quiénes eran los ``partners'' de Huawei en los proyectos de innovación? ¿Cómo los seleccionaban? ¿Cuál es la importancia de un proceso de selección en este tipo de proyectos? }
Los ``partners'' de Huawei son principalmente los profesores universitarios, estudiantes de doctorado y universitarios considerados capaces para realizar el objetivo. Estos se reclutan por medio del programa HIRP y por medio de recomendaciones, usualmente hay muchos candidatos motivados interesados en el proyecto pero se selecciona el mejor basado en su área de relevancia.


\bibliographystyle{plain}
\bibliography{bib.bib} 

%%%%%%%%%%%%%%%%%%%%%%%%%%%%%%%%%%%%%%%%%%%%%%%%%%%%%%%%%%%%%%%%%%%%%%%%%%%%%%%%%%%%%%%%%%%%%%%%%%%%%%%%%%%%%%%%%%%%%%%%%%%%%%%%%%%%%%%
\end{document}

