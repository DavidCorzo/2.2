\documentclass{article}

\usepackage{generalsnips}
\usepackage{calculussnips}
\usepackage{programmingsnips}
\usepackage[margin = 1in]{geometry}
\usepackage{pdfpages}
\usepackage[spanish]{babel}
\usepackage{amsmath}
\usepackage{amsthm}
\usepackage[utf8]{inputenc}
\usepackage{titlesec}
\usepackage{xpatch}
\usepackage{fancyhdr}
\usepackage{tikz}
\usepackage{hyperref}
\usepackage{enumerate}
\title{Ejercicio 5 }
\date{2020 October 27} % , 09:00PM
\author{David Corzo}
\darktheme
\begin{document}
\maketitle
%%%%%%%%%%%%%%%%%%%%%%%%%%%%%%%%%%%%%%%%%%%%%%%%%%%%%%%%%%%%%%%%%%%%%%%%%%%%%%%%%%%%%%%%%%%%%%%%%%%%%%%%%%%%%%%%%%%%%%%%%%%%%%%%%%%%%%%%%%%%%%

\section{Ejercicio \#1}
\inputcode{py}{./EJ5_01_DAVIDCORZO.py}
\begin{enumerate}[(a)]
    \item Si se llegan a completar todos los tiempos.
    \item La ganancia óptima resulta ser: $990$.
    \item Complejidad: la complejidad es $n^2$.
\end{enumerate}

\section{Ejercicio \#2}
\inputcode{py}{./EJ5_02_DAVIDCORZO.py}


%%%%%%%%%%%%%%%%%%%%%%%%%%%%%%%%%%%%%%%%%%%%%%%%%%%%%%%%%%%%%%%%%%%%%%%%%%%%%%%%%%%%%%%%%%%%%%%%%%%%%%%%%%%%%%%%%%%%%%%%%%%%%%%%%%%%%%%%%%%%%%
\end{document}

