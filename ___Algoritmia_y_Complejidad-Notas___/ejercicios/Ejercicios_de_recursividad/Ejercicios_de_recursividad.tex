\documentclass{article}

\usepackage{generalsnips}
\usepackage{calculussnips}
\usepackage[margin = 1in]{geometry}
\usepackage{pdfpages}
\usepackage[spanish]{babel}
\usepackage{amsmath}
\usepackage{amsthm}
\usepackage[utf8]{inputenc}
\usepackage{titlesec}
\usepackage{xpatch}
\usepackage{fancyhdr}
\usepackage{tikz}
\usepackage{hyperref}
\usepackage{minted}
\usepackage{fvextra}
\title{Ejercicios de recursividad}
\date{2020 agosto 03} % , 10:34PM
\author{David Corzo}
\pagestyle{empty}

\fvset{frame=single,numbers=left,numbersep=3pt,breakanywhere=true,breaklines=true}

\begin{document}
\maketitle
%%%%%%%%%%%%%%%%%%%%%%%%%%%%%%%%%%%%%%%%%%%%%%%%%%%%%%%%%%%%%%%%%%%%%%%%%%%%%%%%%%%%%%%%%%%%%%%%%%%%%%%%%%%%%%%%%%%%%%%%%%%%%%%%%%%%%%%%%%%%%%

\section{Librerías}
\begin{minted}[autogobble]{python}
    import time
\end{minted}

\section{Ejercicio $b^n$}
Función recursiva que calcule $b^n$ si $b > 0$, $n >= 0$ pertenece a los números enteros.
\begin{minted}[autogobble]{python}
    def b_n(b: int,n: int):
        if (n == 0):
            return 1
        else: 
            return b * b_n(b, n-1)
\end{minted}

\section{Ejercicio  Fibonacci}
Alguien compra una pareja de conejos. Luego de un mes esos conejos son adultos. Después de dos meses esa pareja de conejos da a luz a otra pareja de conejos. Al tercer mes la primera pareja de conejos da a luz a otra pareja de conejos y sus primeros hijos se vuelven adultos. Caca mes que pasa, cada pareja de conejos adultos da a luz a una nueva pareja de conejos y una pareja de conejos tarda un mes en crecer. Escriba una función recursiva que regrese cuántos conejos adultos se tienen pasados $n$ meses.
\begin{minted}[autogobble]{python}
    def conejos_fib(n):
        if n == 0:
            return 0    
        elif n == 1:
            return 1
        else:
            return conejos_fib(n-2) + conejos_fib(n-1)
\end{minted}

\section{Time performance}
\begin{minted}[autogobble]{python}
    def main():
        start = time.perf_counter()
        b_n(56,9)
        print("{:.50f}".format(time.perf_counter() - start))
        start = time.perf_counter()
        conejos_fib(10)
        print("{:.50f}".format(time.perf_counter() - start))

    if __name__ == "__main__":
        main()

    # output: 0.00000999999999999612310119800895336084067821502686
    # output: 0.00002180000000000237303510175479459576308727264404
\end{minted}

%%%%%%%%%%%%%%%%%%%%%%%%%%%%%%%%%%%%%%%%%%%%%%%%%%%%%%%%%%%%%%%%%%%%%%%%%%%%%%%%%%%%%%%%%%%%%%%%%%%%%%%%%%%%%%%%%%%%%%%%%%%%%%%%%%%%%%%%%%%%%%
\end{document}

