El experimento consta de una serie de pruebas estadísticas hechas a tres poblaciones distintas: una población de personas que tocan el cello, otra que toca la flauta y otra que toca el piano. Las muestras provienen de tomar la medición de dopamina antes y después de tocar el instrumento asignado, puesto a que lo que nos interesa es el aumento o disminución que cause tocar el instrumento se determinó pertinente tomar las diferencias entre la medición posterior a tocar el instrumento y la medición previa a tocarlo.


Pruebas a realizar: 
\begin{itemize}
    \item La primera prueba fue utilizada para comparar los aumentos o disminuciones de las tres poblaciones (Piano, Flauta, Cello) para esto utilizamos la prueba de hipótesis no paramétrica Kruskal-Wallis. (Además condujimos dos pruebas más para dar refuerzo estadístico a esta prueba.)
    \item La segunda prueba fue utilizada para analizar si existe una diferencia estadística entre los niveles de dopamina entre las poblaciones que tocan el piano y los que tocan el cello. Mann-Whitney-Wilcoxon y comparar si estos resultados son congruentes a la prueba anterior (Kruskal-Wallis).
    \item La tercera prueba fue realizada para analizar si existe diferencia estadísticamente significativa entre las diferencias de los niveles de dopamina antes y después de tocar el instrumento respectivo. Para esta prueba se utilizaron las poblaciones que tocan Cello y Piano y se determinó pertinente usar la prueba Mann-Whitney-Wilcoxon, además comparar si los resultados provenientes de esta prueba apoyaban o no a los resultados concluidos en la primera prueba (Kruskal-Wallis).
\end{itemize}
