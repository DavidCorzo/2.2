% \subsection{Marco teórico}
Tocar un instrumento tiene muchos beneficios, entre ellos, el empezar a hacerlo a una corta edad puede ayudar en la memoria, cognición y otros beneficios en un futuro [1]. La dopamina, es uno de varios neurotransmisores que utilizan las neuronas para comunicarse entre ellas. La dopamina es considerada como la causante de sensaciones placenteras y de relajación. El cerebro libera dopamina cuando una persona escucha música que le parece agradable e incluso, cuando sabe que la escuchará en el futuro cercano [2]. La música contribuye a disminuir la ansiedad, mejorar la frecuencia cardiaca y el humor, por estas y más ventajas, la música se usa cada vez más como herramienta terapéutica [2]. 
Como hemos mencionado antes, si al escuchar música se producen efectos positivos en nuestro cuerpo, el interpretar música o tocar un instrumento producen aún más, ya que al tocar, no solo se está escuchando música sino que también se pone en acción la coordinación de la mente y el cuerpo.

La música es considerada uno de los elementos que causan más placer en la vida ya que, como hemos mencionado anteriormente, ayuda a estabilizar las emociones debido a la liberación de dopamina y otros neurotransmisores. Otra función a cargo de la dopamina es regular la captación de la información, en otras palabras, los recuerdos y la memoria [3]. Si la información que entra a nuestro cerebro no nos gusta, el hipocampo no se activa y el recuerdo no se almacena en la memoria, es por eso que los músicos tienden a tener una mejor memoria [3]. El hipocampo es una de las partes del cerebro que está situado en el sistema límbico. Está relacionado tanto con los procesos mentales relacionados con la memoria como con los procesos de producción y regulación de estados emocionales. El hipocampo permite que la información pase a la memoria a largo plazo conectando la información recibida con valores positivos o negativos dependiendo de si los recuerdos son placenteros o dolorosos [3].

La música es procesada por ambos hemisferios. El hemisferio derecho recibe el estímulo musical y el hemisferio izquierdo interpreta y controla la información. En conclusión, tocar un instrumento estimula el cerebro de tal manera que es capaz de poner a trabajar ambos lados o hemisferios del cerebro y liberar sustancias que ayudan a liberar sensaciones negativas y reemplazar esas sensaciones por placer y felicidad [3].


\vspace{0.5cm}
%----------------------------------------------------------------------------------------
% \subsection{Relevancia del estudio}
Este estudio es de gran relevancia ya que al obtener resultados e información sobre el efecto causado en el aumento o disminución de la dopamina al tocar un instrumento, se pueden hacer recomendaciones a instituciones dedicadas a la salud emocional y a personas con necesidad de mejorar su estado de ánimo, estrés y humor. Incluso, con los resultados posibles de este estudio, se pueden recomendar instrumentos en específico, con indicaciones sobre qué instrumento aumenta más la secreción de dopamina al tocarlo.



\vspace{0.5cm}
%----------------------------------------------------------------------------------------
% \subsection{Objetivos}
El objetivo principal de la investigación fue tomar muestras de dopamina entre personas seleccionadas aleatoriamente antes y después de tocar tres tipos de instrumentos, esto para demostrar si existe o no una diferencia estadísticamente significativa en el aumento de dopamina  al  tocar piano, cello o flauta.


\vspace{0.5cm}
%----------------------------------------------------------------------------------------
% \subsection{Hipótesis de investigación}
Se analizó tres diferentes instrumentos para comprobar si existe diferencia estadísticamente significativa entre el nivel de dopamina que segrega cada una de ellos, y determinamos como hipótesis nula la proposición: ``No existe diferencia estadísticamente significativa  en el aumento de dopamina entre tocar piano, cello o flauta''; y como hipótesis alternativa la proposición que ``Sí existe diferencia estadísticamente significativa  en el aumento de dopamina entre tocar el piano, cello o flauta''. 

\vspace{0.5cm}
%----------------------------------------------------------------------------------------
% \subsection{Justificación del proyecto}
Con el marco teórico previo, se puede afirmar que tocar música tiende a aumentar el nivel de dopamina en el cerebro de un individuo. De aquí, nace la interrogante de si el instrumento que se toca afecta en alguna manera la cantidad producida. Por esto, se pretende con esta investigación evaluar si existe una diferencia estadísticamente significativa entre los niveles de dopamina que generan los tres instrumentos analizados.


%----------------------------------------------------------------------------------------
