Tomando en consideración los resultados del proyecto de investigación y las conclusiones, se recomienda investigar otros aspectos relacionados con este tema:  

\begin{itemize}
    \item Cambios en los niveles de dopamina después de escuchar distintos géneros de música.
    \item Diferencia estadística de otros factores fisiológicos después de tocar los instrumentos más utilizados en el mundo de la música.
    \item Cual es el instrumento que estadísticamente causa una segregación más grande de dopamina.
    \item Cambios en niveles de otros neurotrasmisores después de tocar un instrumento.
\end{itemize}

A la hora de realizar el experimento y el trabajo de investigación, se recomienda fijar un rango de edades para los sujetos y realizar las pruebas alrededor de la misma hora del día, de forma de evitar sesgos externos en los resultados. Aunque no es necesaria la normalidad para la mayoría de pruebas, se recomienda que las muestras de todas las poblaciones sean $n\geq 7$ ($n_i\geq 5$ para Kruskal-Wallis \& $n\geq 7$ para Mann-Whitney-Wilcoxon y que las muestras sean independientes) de forma que se pueda usar mayor variedad de pruebas no paramétricas. 


Por último se recomienda realizar el procedimiento estadístico apoyándose de las herramientas de Excel y sus fórmulas, así como de Geogebra. Especialmente utilizar esta última para la generación rápida y fácil de distintas gráficas de los datos, como gráficas q-q, diagramas de caja e histogramas.
