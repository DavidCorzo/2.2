
Usando los supuestos de no normalidad, y el de una significancia de 0.05, podemos analizar lo siguiente:
\begin{itemize}
    \item La prueba Kruskal-Wallis confirmó que en efecto las muestras de poblaciones de piano, cello y flauta son diferentes. Puesto a que la prueba Kruskal-Wallis solamente nos confirmó que en efecto una o más poblaciones difieren una respecto de la otra. Para dar refuerzo estadístico a la prueba Kruskal-Wallis, decidimos observar algunos casos en los que la $H_0$ no se pueda rechazar y otros en los que sí. Utilizando la prueba Mann-Whitney-Wilcoxon comparamos el caso piano-cello, en el cual pudimos rechazar la $H_0$ y por ende afirmar que las muestras de piano y de cello no provenían de la misma población, afirmando entonces que había una diferencia estadística entre la dopamina segregada al tocar cello y piano; comparamos el caso cello-flauta, en el cual no pudimos rechazar la $H_0$  y por ende no se pudo afirmar que las poblaciones son diferentes.
    \item Las dos pruebas Mann-Whitney-Wilcoxon se hicieron como refuerzo estadístico para sustanciar los resultados observados en la prueba Kruskal-Wallis. En suma, las pruebas demuestran, y están en alineación una respecto de la otra, que las poblaciones difieren de una manera estadísticamente significativa y por lo tanto no se segrega el mismo nivel de dopamina al tocar algunos instrumentos respecto de los otros.
\end{itemize}
