Tomando en cuenta el análisis de la base de datos y muestras de las poblaciones, se pudo realizar la prueba Kruskal-Wallis. Después de plantear las hipótesis, logramos rechazar la hipótesis nula y afirmar que las poblaciones son diferentes. A parte, pudimos determinar que las poblaciones de piano y cello son diferentes, así como pudimos observar que no tuvimos suficiente evidencia para decir que la población de cello y flauta son diferentes. Esto último tal vez podría tener un resultado distinto si se obtiene una muestra de mayor tamaño para estas dos poblaciones en específico.


Intuimos que estas diferencias entre las poblaciones se deben a las diferentes complejidades asociadas con tocar cada instrumento, considerando que los propios músicos están de acuerdo que hay instrumentos más difíciles que otros. Por lo que estas diferencias en dificultad pudieron haber causado las diferencias en los aumentos de dopamina. 



Puesto a estos resultados podría abrirse una nueva interrogante que plantee resolver qué instrumentos son los que estadísticamente tienden a segregar la mayor cantidad de dopamina posterior al ser tocados y cuales tienen la desventaja estadística de no segregar tanto, esta interrogante va más allá del enfoque de esta investigación por lo que podría tratarse en futuras investigaciones.
