\documentclass[a4paper]{report}

\usepackage{generalsnips}
\usepackage{calculussnips}
\usepackage[margin = 0.80in]{geometry}
\usepackage{pdfpages}
\usepackage[spanish]{babel}
\usepackage{amsmath}
\usepackage{amsthm}
\usepackage[utf8]{inputenc}
\usepackage{titlesec}
\usepackage{xpatch}
\usepackage{fancyhdr}
\usepackage{tikz}
\usepackage{hyperref}
\usepackage{float}
\decimalpoint
\begin{document}
\begin{titlepage}
    \begin{center}
        \vspace*{1cm}
 
        {\Huge\textbf{Investigación del efecto de consumo de cannabis y alcohol de jóvenes de las edades de 18-25 años}}
 
        \vspace{0.5cm}
        % \pregunta{El canabis afecta negativamente el pensamiento} 
             
        % \vspace{1.5cm}
 
        {Anesveth Maatens (20190339), Andrea Reyes (20190265), David Corzo (20190432), Daniel Cabrera (20190069) y Fabricio Juárez (20190361)}
 
        \vfill
             
        \vspace{0.8cm}
        
        
        \includegraphics[width=0.5\textwidth]{ufmlogo.png} \\ 
        Facultad de Ciencias Económicas \\
        Universidad Francisco Marroquín \\
        Guatemala \\
        25 de agosto de 2020
             
    \end{center}
\end{titlepage}

%%%%%%%%%%%%%%%%%%%%%%%%%%%%%%%%%%%%%%%%%%%%%%%%%%%%%%%%%%%%%%%%%%%%%%%%%%%
\section{Descripción de la investigación}
De un estudio fueron obtenidos resultados de pruebas matemáticas después de ingerir alcohol y después de ingerir cannabis.
La hipótesis de investigación indica que el cannabis afecta negativamente en un nivel mayor el desempeño mental en comparación con el alcohol. A continuación se presentan las puntuaciones obtenidas en el examen matemático después de ingerir alcohol y después de ingerir cannabis.

\section{Prueba de hipótesis}

\begin{itemize}
    \item \textbf{Hipótesis nula:} El alcohol afecta negativamente de mayor manera el desempeño mental que el cannabis.
    \item \textbf{Hipótesis alternativa:} El cannabis afecta negativamente de mayor manera el desempeño mental que el alcohol.
\end{itemize}

\begin{center}
    {
        \begin{tabular}{ |l|c| }
            \hline
                \multicolumn{2}{|c|}{\textbf{Prueba matemática después de cannabis}} 
            \\ 
            \hline
                Media & 29.5 \\ 
            \hline
                Desviación estándar & 6.606135454 \\ 
            \hline
                Varianza de la muestra & 43.64102564 \\ 
            \hline
                $n_1$ & 40 \\ 
            \hline
        \end{tabular}
    }
    {
        \begin{tabular}{ |l|c| }
            \hline
                \multicolumn{2}{|c|}{\textbf{Prueba matemática después de alcohol}} 
            \\ 
            \hline
                Media & 26.55 \\ 
            \hline
                Desviación estándar & 7.316822906 \\ 
            \hline
                Varianza de la muestra & 53.53589744 \\ 
            \hline
                $n_2$ & 40 \\ 
            \hline
        \end{tabular}
    }
\end{center}

\begin{enumerate}
    \item \textbf{Establecer el parámetro de interés:} $\mu_1-\mu_2$ 
    \item \textbf{Establecer hipótesis:}
        \begin{center}
           \begin{align*}
               H_0: \mu_1 \geq \mu_2 \\ 
               H_a: \mu_1 < \mu_2 \\
           \end{align*}
        \end{center}
    
    \item \textbf{Establecer significancia:} 0.05
    \item \textbf{Estadístico de prueba:}
        Tomando como referencia la fórmula de distribución T: 
        \[
          t = \frac{
              \bar{x_1}-\bar{x_2}-D_0
            }{
                \sqrt{
                    \cfrac{s_1^2}{n_1} + \cfrac{s_2^2}{n_2}
                }
            }
        \] Sustituyendo los datos: 
        \begin{center}
           \begin{align*}
                t = \frac{
                    29.5-26.55-0
                }{
                    \sqrt{
                        \cfrac{43.64102564}{40} + \cfrac{53.53589744}{40}
                    }
                }
           \end{align*}
        \end{center}
\end{enumerate}

\begin{figure}[H]
    \centering
    \includegraphics[width=0.5\textwidth]{first_attempt}
\end{figure}


%%%%%%%%%%%%%%%%%%%%%%%%%%%%%%%%%%%%%%%%%%%%%%%%%%%%%%%%%%%%%%%%%%%%%%%%%%%
\end{document}

