\documentclass{article}

\usepackage{generalsnips}
\usepackage{calculussnips}
\usepackage[margin = 1in]{geometry}
\usepackage[spanish]{babel}
\usepackage{amsmath}
\usepackage{amsthm}
\usepackage[utf8]{inputenc}
\usepackage{hyperref}
\title{4 valores y 12 principios del manifiesto ágil}
\date{2020 August 16} % , 07:30PM
\author{David Corzo}
\begin{document}
\maketitle
%%%%%%%%%%%%%%%%%%%%%%%%%%%%%%%%%%%%%%%%%%%%%%%%%%%%%%%%%%%%%%%%%%%%%%%%%%%%%%%%%%%%%%%%%%%%%%%%%%%%%%%%%%%%%%%%%%

El manifiesto ágil fue creado en el 2001 por 17 críticos de modelos de procesos. Estos planeaban consensuar cuales eran los mayores valores y principios del que agilizaba el desarrollo de software.

\section{4 Valores}
\begin{enumerate}
    \item Valorar más a los individuos y su interacción a los procesos y las herramientas
        \begin{itemize}
            \item Adaptar los procesos a las personas, se favorece la auto-organización de equipos.
            \item Los procesos requieren talento y necesitan.
            \item Se intenta que el resultado del proceso sea de alta calidad por el proceso y no por los que lo ejecuten. 
        \end{itemize}
    \item Valorar más el software funcionando que la documentación exhaustiva.
        \begin{itemize}
            \item Se pierde el tiempo con la documentación exhaustiva.
            \item La documentación en si no es el problema, es la documentación innecesaria. 
        \end{itemize}
    \item Valorar más la colaboración con el cliente que la negociación contractual.
        \begin{itemize}
            \item Los proyectos se van detallando conforme se desarrolla y no todo al principio. 
            \item El objetivo es proporcionar el mayor valor posible al cliente. 
            \item Cosas y detalles cambian por: velocidad de cambio en el entorno del cliente.
        \end{itemize}
    \item Valorar más la respuesta ante el cambio que seguir un plan.
        \begin{itemize}
            \item Para elaborar productos cambiantes es más valiosa la velocidad de respuesta que la de seguimiento y aseguramiento de planes. 
            \item Valores de la gestión ágil: anticipación y adaptación. Diferente a los valores de gestión de proyectos: planificación y control, evitar desviarse del plan. 
        \end{itemize}
\end{enumerate}


\section{12 principios}
\begin{enumerate}
    \item Nuestra principal prioridad es satisfacer al cliente a través de la entrega temprana y continua de software de valor.
    \item Son bienvenidos los requisitos cambiantes, incluso si llegan tarde al desarrollo. Los procesos ágiles se doblegan al cambio como ventaja competitiva para el cliente.
    \item Entregar con frecuencia software que funcione, en periodos de un par de semanas hasta un par de meses, con preferencia en los periodos breves.
    \item Las personas del negocio y los desarrolladores deben trabajar juntos de forma cotidiana a través del proyecto.
    \item Construcción de proyectos en torno a individuos motivados, dándoles la oportunidad y el respaldo que necesitan y procurándoles confianza para que realicen la tarea.
    \item La forma más eficiente y efectiva de comunicar información de ida y vuelta dentro de un equipo de desarrollo es mediante la conversación cara a cara.
    \item El software que funciona es la principal medida del progreso.
    \item Los procesos ágiles promueven el desarrollo sostenido. Los patrocinadores, desarrolladores y usuarios deben mantener un ritmo constante de forma indefinida.
    \item La atención continua a la excelencia técnica enaltece la agilidad.
    \item La simplicidad como arte de maximizar la cantidad de trabajo que se hace, es esencial.
    \item Las mejores arquitecturas, requisitos y diseños emergen de equipos que se auto-organizan.
    \item En intervalos regulares, el equipo reflexiona sobre la forma de ser más efectivo y ajusta su conducta en consecuencia.
\end{enumerate}

\section*{Referencias}
\begin{enumerate}
    \item \url{https://www.scrummanager.net/bok/index.php?title=El_manifiesto_%C3%A1gil}
    \item \url{https://medium.com/@agnostic/manifiesto-agile-y-sus-4-principios-aplicados-al-desarrollo-a456b4ac9a38}
\end{enumerate}


%%%%%%%%%%%%%%%%%%%%%%%%%%%%%%%%%%%%%%%%%%%%%%%%%%%%%%%%%%%%%%%%%%%%%%%%%%%%%%%%%%%%%%%%%%%%%%%%%%%%%%%%%%%%%%%%%%
\end{document}

