\begin{itemize}
    \item El primer gran filosofo fue descartes,  lo que terminó siendo Kant y Hegel viene del legado de Descartes, la filosofía idealista alemán viene de descartes. 
\end{itemize}

\subsection{Descartes y su legado}
\begin{itemize}
    \item El renacimiento fue una crítica (superficial), el espíritu del renacimiento es un espíritu crítico, Descartes introduce un cambio crucial, introdujo muchos otros cambios, así como Stephen Hawking cambió muchas cosas, en su libro ``and God created the integers'' dijo ``olviden los aportes de descartes a la filosofía sus más grandes aportes son en la matemática''. 
    \item La muerte de Descartes está envuelta en misterio, era maestro de la reina Cristina de Suecia. 
    \item Uno de los aportes más grandes de Descartes fue la geometría analítica, esto fundaron bases para los aportes de Leibniz \& Newton. 
    \item Descartes tenía a un profesor de matemática que le decían el nuevo Euclides. 
    \item Las meditaciones metafísicas de Descartes eran fuertemente influenciadas a las escrituras de San Ignacio de Loyola. 
\end{itemize}

\subsection{La sustancia}
\begin{itemize}
    \item Descartes rompe con la sustancia, introduce la duda, introduce el escepticismo, pregunta \pregunta{cómo sé lo que sé} Las únicas dos preguntas filosóficas es \pregunta{como sabes que lo que sabes es verdadero} y \pregunta{what do you mean}.
    \item Descartes quería separar la ciencia de la filosofía, y la filosofía de la teología. 
    \item Descartes no era tanto un científico empírico como Newton, Descartes era un matemático de tiempo completo. 
    \item Él decía que la filosofía no es una ciencia por que no tiene método. 
\end{itemize}

\subsection{La división de la sustancia}
\begin{itemize}
    \item Descartes divide la sustancia, decían que la sustancia era una, dice que la res extensa y la res cogitans era totalmente diferente, lo general era diferente a lo particular. 
    \item Descartes divide la sustancia en pensamiento y la realidad.
    \item La consecuencia de esta división, es que la ciencia es la metodología para proyectar en el mundo real las ideas de la mente. 
    \item Entonces ese radicalismo, o en filosofía, \emph{dualismo cartesiano} es 
\end{itemize}

\subsection{Deus ex machina}
\begin{itemize}
    \item Significa un dios fuera de máquina.
    \item La única garantía que tenemos para probar la existencia de las cosas, en el sentido de ``pienso luego existo'', es Dios. 
\end{itemize}

\section{Spinoza y la axiomatización geométrica de la sustancia}
\begin{itemize}
    \item La primera crítica a Descartes la hizo Espinoza, en su libro ``ética'' es el intento 
    \item Espinoza por criticar a la iglesia ortodoxa judía le aplican la pena correspondiente de ley del hielo perpetua, mientras él vivía en el gueto.
    \item ``Devolver la idea al mundo, devolver el mundo al sujeto'', como Descartes decía que todo es una proyección y desconfiar en todo, Espinoza consideraba que Descartes le había robado el mundo al sujeto, Espinoza deseaba devolvérselo al individuo.
\end{itemize}

\subsection{Leibniz}
\begin{itemize}
    \item Probablemente el único filósofo comparable con Descartes es Leibniz. 
\end{itemize}

%----------------------------------------------------------------------------------------

