\section{Introducción}
\begin{itemize}
    \item Para Aristóteles la lógica y la metafísica son lo mismo.
    \item Descartes rompió con el centro de esta idea, la sustancia. 
    \item La tecnología es el producto material de lo subjetivo. 
    \item Los filósofos religioso pensaban que la metafísica había hecho mucho bien. 
    \item Descartes se sentía mal en Francia por el tradicionalismo. Descartes entonces se fue a Holanda (por lo liberal que es), Holanda es un safe-heaven para personas que buscan la libertad. 
        \begin{itemize}
            \item En esa época Holanda era el Nueva York de hoy. Anteriormente Venecia. NY en el término de que ahí se hace todas las actividades bancarias más grandes. 
            \item Este fue el lugar donde Descartes fue buscando libertad. Y los holandeses lo recibieron con brazos abiertos, Holanda no tenía filósofos de sí mismos. 
            \item Los holandeses metieron la filosofía y la matemática se metieron a los \emph{pénsums} del sistema educativo. Después de un tiempo fue acusado de pensar agnósticamente por pastores protestantes.
        \end{itemize}

    \item Filósofos como Espinoza, Leibniz, Kant y Hume buscaban reconciliar la ciencia con la religión.
\end{itemize}

\section{Leibniz}
\begin{itemize}
    \item Cálculo, no los únicos aportes de Leibniz, también hay filosofía de Leibniz. 
    \item Era un filósofo multifacético, apasionado por la matemática, derecho, filosofía etc. 
    \item Trabajaba mucho, cuando le dan 15 días de vacaciones escribe la Mónada.
    \item Recordar que Descartes junta geometría plana euclidiana y el álgebra, de esto sale la geometría analítica. 
    \item El cálculo infinitesimal, Leibniz: ``Cómo es posible que lo que tiene medida esté hecho de lo que no tiene medida''.
    \item La mónada cumple en la metafísica la función lo que el punto cumple en el cálculo infinitesimal.
    \item La interpretación: la monada es las cosas que tienen medida, que podemos tocar, ver, etc, están hechas de un principio subjetivo, por que cuando descomponemos lo real en sus partes indivisibles resulta que solamente se puede pensar y lo que se puede pensar 
        \begin{itemize}
            \item Esta propuesta de Leibniz es la respuesta a los Res Extensa y Res Cogitans, características de dioses cristianos, judíos, etc. 
            \item Un mundo sin Dios es absurdo, que la materia solo se tenga que aceptar. 
            \item No podemos elaborar el edificio de las cosas basado en una suposición. 
        \end{itemize}
    
    \item Leibniz era tan matemático como Descartes, no se puede alegar ignorancia de parte de Leibniz por la parte de la matemática. 
        \begin{itemize}
            \item Leibniz une lo racional y lo material. 
            \item No es como Hume, Kant, Espinoza que estaban consientes de la matemática y física pero no eran profesionales.
        \end{itemize}
    
    \item El retorno de la idea al mundo. 
    \item Leibniz tenía una vocación idealista, por que habla de un lado subjetivo.
\end{itemize}

\section{Empirismo británico}
\begin{itemize}
    \item John Locke y Hume, los ingleses intentan unir lo ideal con lo racional, su manera fue desde los objetos. 
    \item La imagen como la mediación entre el mundo sensible y el trabajo del intelecto. La primera cosa que sucede en nuestra mente es el concepto de la imagen, la imagen es la mediación entre el mundo de la experiencia sensible y el mundo de la experiencia intelectual. 
    \item Se intenta describir como todo lo que existe ha pasado por los sentidos, para Hume los ejemplos que tenía, en la mente no hay nada que no haya pasado por los sentidos. 
        \begin{itemize}
            \item Los videojuegos, Hume diría, cada parte de los monstruos están basados en formas que existen, siempre son de cosas que se han experimentado. 
        \end{itemize}
    \item La idea que por mucha abstracción se vuelve posible por que hemos tenido una experiencia sensible, visualizamos esto por medio de la imagen en la mente, una impresión o huella. 
        \begin{itemize}
            \item Huella se refiere a algo similar a cuando se hace un molde para hacer una escultura. La imagen se vuelve algo reproducible, a través del tiempo se va deteriorando, similar a cuando un músico recuerda una pieza, a través del tiempo se van olvidando y deteriorando. 
        \end{itemize}
    \item El concepto de la imagen cumple la misma función que la mónada de Leibniz.
    \item Recordar que el empirismo está en contra de la metafísica.
\end{itemize}


\section{Idealismo alemán: una síntesis crítica}
\begin{itemize}
    \item No son todos los autores, son solo los necesarios para entender a Hegel. 
    \item Kant, alguien aislado como Aristóteles, no por las mismas razones pero aislado. 
\end{itemize}

\section{Kant: filosofía crítica y los límites del pensamiento}
Kant era brillante. 
\begin{enumerate}
    \item Las tres críticas eran como tres momentos de una misma obra. 
    \item Primera critica (crítica de la razón pura): formas a priori de la sensibilidad. 
        \begin{itemize}
            \item Primeras 100 páginas expone la parte de las formas a priori de la sensibilidad.
            \item Hay conceptos que son categorías, pero que están vinculadas a la experiencia sensible. 
            \item Cuando dice ``el tiempo y el espacio'', el tiempo solo es empírico, el tiempo surge de la experiencia sensible; lo mismo con el espacio, como condición espacial para que existan y se manifiesten objetos. Conceptos un poco lo de Leibniz de la Mónada.
            \item Esta es la idea de la sustancia de Aristóteles. 
            \item La filosofía para Kant tenía que tener un fundamento crítico, no se trata de volver a Santo Tomás y Aristóteles, es de recuperar una tradición después de un trauma (Descartes). 
            \item La metafísica tradicional necesita antes de una noción crítica. 
            \item Kant intenta mantener un equilibrio entre lo ideal y lo material.
            \item A propósito de un diálogo con Newton, Kant hace este libro para decir y ejemplificar que la razón tiene límites. 
            \item \pregunta{Cuál es la pregunta que da origen a la respuesta de la ciencia} 
            \item Cuando tomamos los problemas como temas y no como preguntas empezamos a repetir lo que se ha dicho. 
        \end{itemize}
    \item Segunda critica (crítica de la razón práctica): el Faktum moral.
        \begin{itemize}
            \item Tiene relación con la primera, abre la pregunta ``\pregunta{hay algo que el ser humano pueda conocer absolutamente}'', ej. no tenemos conocimiento absoluto. 
            \item La ciencia natural entonces tiene límites. 
            \item Si todo el conocimiento es limitado entonces el hombre no puede conocer la verdad, se imagina la verdad.
            \item Kant: Lo único que podemos conocer absolutamente es a nosotros mismos, esto es un consenso con Descartes.
            \item Yo: esto es una computadora, Kant: yo pienso que esto es una computadora, toda la consciencia es un pensamiento, esto es el Faktum moral. 
            \item Esta es la metafísica de Kant, es la primera obra que junta de manera sistemáticas la metafísica y la ética, en el mundo moderno, anteriormente Aristóteles.
        \end{itemize}
    \item Tercera critica (crítica del juicio): estética y filosofía de la naturaleza. 
        \begin{itemize}
            \item En el sentido griego del término estético.
                \begin{itemize}
                    \item Hoy lo entendemos como filosofía del arte. 
                    \item \emph{Estetos} es lo sensible. 
                    \item O sea dos formas distintas de referirse a lo natural, 1. la ciencia, y 2. el arte.
                \end{itemize}
            \item Este no es muy importante para este curso. 
            \item 
        \end{itemize}
    \item El último Kant: la historia y la religión: pregunta de Kant: \pregunta{qué esperar} en este y el otro mundo.
        \begin{itemize}
            \item Este último en el que fue obligado a opinar, dado al conflicto de las guerras napoleónicas. 
            \item Dos temas que no había tocado: historia, religión. 
            \item Él no quería repetir las cosas que ya se habían dicho. La religión consiste de repetir lo que se ha dicho. 
            \item Kant: \pregunta{cual es la pregunta que vuelve la religión una pregunta científica}
            \item Kant: la religión no es solo preocupación de sacerdotes y teólogos.  
            \item Kant: \pregunta{que podemos esperar en este mundo} política, leyes; \pregunta{que podemos esperar en el otro mundo} religión. 
            \item Por el uso de la palabra \emph{religión} se puede intuir que para él la religión es un fenómeno cultural, a diferencia de la ciencia la religión no tiene un mecanismo de prueba científica que pruebe algo, a diferencia de Aquino que intentaba presentar pruebas de Dios mediante lógica y ciencia, Kant decía que no hay pruebas que prueben la existencia de Dios, la religión es un postulado de la razón. Así como no existe una prueba de Dios no se puede probar la no existencia del mismo. 
        \end{itemize}
\end{enumerate}
\begin{itemize}
    \item Kant nunca fue perseguido, fue respetado, a diferencia de Espinoza que sí fue perseguido. 
    \item Kant era de origen escocés, cuando llegó a Alemania se cambió el nombre. 
    \item Kant no escribió durante la mayoría de su vida, cuando él tenía como 60 años él empezó a escribir sus obras serias.
    \item En aquella época tener mucho conocimiento de la época era muy común, hoy en día el conocimiento es tan especializado que es imposible estar consiente de los avances de la ciencia, en aquella época ni siquiera se podía observar el átomo, lo que era conocido era mucho menor que hoy en día.
    \item Hoy en día tenemos ciencia que explica diferentes niveles de la experiencia, está el subatómico y la teoría de cuerdas.
    \item Desde Descartes hasta Hegel la filosofía se dedica a recuperar la unidad de la sustancia, del dualismo al monismo.
\end{itemize}


\section{Hegel}
\begin{enumerate}
    \item La universidad de Jena: Schelling, Hölderlin y Hegel.
        \begin{itemize}
            \item Las universidades alemanas habían cambiado mucho. 
            \item Napoleón toma la universidad de París, se la arrancó a los dominicos. En la USAC también pasó esta revolución. Las estructuras universitarias de facultades son napoleónicas, las estructuras anglosajonas están estructurados por departamentos este es el modelo alemán.
            \item La relación de Hegel y Schelling era difícil, Hölderlin es un gran poeta alemán. Hegel y Schelling eran amigos al principio, después se separaron, ellos fueron los primeros en hacer una \emph{carrera académica} (escalar en las universidades más prestigiosas para buscar posiciones para dar clases) Schelling y Hegel se separaron por que hicieron carreras académicas. Para trabajar dando clase, el ministro de educación te tiene que aprobar, los alemanes tienen muchos ministros de educación, un ministro de educación 
            \item Cuando Hegel muere de colera Schelling (quien se había distanciado mucho y era ya un crítico del pensamiento de Hegel) toma la cátedra que Hegel había dejado vacante. 
            \item Hegel y Hölderlin: Hölderlin era el filósofo más profundo de todos,, se puede decir, que se tiene que cambiar la poesía $\rightarrow$ las ideas más originales de la ``Fenomenología del espíritu''  son ideas que Hölderlin había tenido en sus años universitarios de Jena debido a su grande intuición filosófica que él tenía. 
            \item La idea del Espíritu es una que si vamos para atrás de Hegel a Hölderlin, al \emph{Sturm un drag}. 
            \item Hilo conductor de esto es Aristóteles. Existe un vínculo entre Hölderlin y Aristóteles en términos de la noción del espíritu.
            \item Schelling, Hölderlin y Hegel querían ser pastores protestantes. 
        \end{itemize}
    
    \item Ensayos de juventud: modernidad y cristianismo. Responder a la pregunta: \pregunta{puede el cristianismo ser una ``religión moderna''} 
        \begin{itemize}
            \item \pregunta{Puede el cristianismo ser una religión moderna}, Kant se percibe como negativo hacia la religión. 
            \item Para Schelling, Hölderlin y Hegel no era fácil rechazar la religión, estos todos tenían un maestro Gottfried. 
            \item La experiencia del Faktum moral de Kant es lo que hace que todas las personas pueden identificarse, esto dio inicios a los derechos humanos. 
            \item La idea de un gobierno universal, basado en que todos somos iguales, todos somos Faktum moral. 
            \item La política actual es un debate entre Hegel y Kant, Kant que defendería el gobierno global y Hegel defendiendo la soberanía  
        \end{itemize}
        \begin{enumerate}
            \item ``Cristiandad, vida moderna y el ideal de belleza: el espíritu de la cristiandad'' 
            \item ``El espíritu del cristianismo y su destino''
            \item ``Fe y conocimiento''
        \end{enumerate}
\end{enumerate}
