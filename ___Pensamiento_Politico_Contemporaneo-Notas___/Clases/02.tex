\section{Introducción}
\begin{itemize}
    \item Para Aristóteles la lógica y la metafísica son lo mismo.
    \item Descartes rompió con el centro de esta idea, la sustancia. 
    \item La tecnología es el producto material de lo subjetivo. 
    \item Los filósofos religioso pensaban que la metafísica había hecho mucho bien. 
    \item Descartes se sentía mal en Francia por el tradicionalismo. Descartes entonces se fue a Holanda (por lo liberal que es), Holanda es un safe-heaven para personas que buscan la libertad. 
        \begin{itemize}
            \item En esa época Holanda era el Nueva York de hoy. Anteriormente Venecia. NY en el término de que ahí se hace todas las actividades bancarias más grandes. 
            \item Este fue el lugar donde Descartes fue buscando libertad. Y los holandeses lo recibieron con brazos abiertos, Holanda no tenía filósofos de sí mismos. 
            \item Los holandeses metieron la filosofía y la matemática se metieron a los \emph{pénsums} del sistema educativo. Después de un tiempo fue acusado de pensar agnósticamente por pastores protestantes.
        \end{itemize}

    \item Filósofos como Espinoza, Leibniz, Kant y Hume buscaban reconciliar la ciencia con la religión.
\end{itemize}

\section{Leibniz}
\begin{itemize}
    \item Cálculo, no los únicos aportes de Leibniz, también hay filosofía de Leibniz. 
    \item Era un filósofo multifacético, apasionado por la matemática, derecho, filosofía etc. 
    \item Trabajaba mucho, cuando le dan 15 días de vacaciones escribe la Mónada.
    \item Recordar que Descartes junta geometría plana euclidiana y el álgebra, de esto sale la geometría analítica. 
    \item El cálculo infinitesimal, Leibniz: ``Cómo es posible que lo que tiene medida esté hecho de lo que no tiene medida''.
    \item La mónada cumple en la metafísica la función lo que el punto cumple en el cálculo infinitesimal.
    \item La interpretación: la monada es las cosas que tienen medida, que podemos tocar, ver, etc, están hechas de un principio subjetivo, por que cuando descomponemos lo real en sus partes indivisibles resulta que solamente se puede pensar y lo que se puede pensar 
        \begin{itemize}
            \item Esta propuesta de Leibniz es la respuesta a los Res Extensa y Res Cogitans, características de dioses cristianos, judíos, etc. 
            \item Un mundo sin Dios es absurdo, que la materia solo se tenga que aceptar. 
            \item No podemos elaborar el edificio de las cosas basado en una suposición. 
        \end{itemize}
    
    \item Leibniz era tan matemático como Descartes, no se puede alegar ignorancia de parte de Leibniz por la parte de la matemática. 
        \begin{itemize}
            \item Leibniz une lo racional y lo material. 
            \item No es como Hume, Kant, Espinoza que estaban consientes de la matemática y física pero no eran profesionales.
        \end{itemize}
    
    \item El retorno de la idea al mundo. 
    \item Leibniz tenía una vocación idealista, por que habla de un lado subjetivo.
\end{itemize}

\section{Empirismo británico}
\begin{itemize}
    \item John Locke y Hume, los ingleses intentan unir lo ideal con lo racional, su manera fue desde los objetos. 
    \item La imagen como la mediación entre el mundo sensible y el trabajo del intelecto. La primera cosa que sucede en nuestra mente es el concepto de la imagen, la imagen es la mediación entre el mundo de la experiencia sensible y el mundo de la experiencia intelectual. 
    \item Se intenta describir como todo lo que existe ha pasado por los sentidos, para Hume los ejemplos que tenía, en la mente no hay nada que no haya pasado por los sentidos. 
        \begin{itemize}
            \item Los videojuegos, Hume diría, cada parte de los monstruos están basados en formas que existen, siempre son de cosas que se han experimentado. 
        \end{itemize}
    \item La idea que por mucha abstracción se vuelve posible por que hemos tenido una experiencia sensible, visualizamos esto por medio de la imagen en la mente, una impresión o huella. 
        \begin{itemize}
            \item Huella se refiere a algo similar a cuando se hace un molde para hacer una escultura. La imagen se vuelve algo reproducible, a través del tiempo se va deteriorando, similar a cuando un músico recuerda una pieza, a través del tiempo se van olvidando y deteriorando. 
        \end{itemize}
    \item El concepto de la imagen cumple la misma función que la mónada de Leibniz.
    \item Recordar que el empirismo está en contra de la metafísica.
\end{itemize}


\section{Idealismo alemán: una síntesis crítica}
\begin{itemize}
    \item No son todos los autores, son solo los necesarios para entender a Hegel. 
    \item Kant, alguien aislado como Aristóteles, no por las mismas razones pero aislado. 
\end{itemize}

\section{Kant: filosofía crítica y los límites del pensamiento}
Kant era brillante. 
\begin{enumerate}
    \item Primera critica (crítica de la razón pura): formas a priori de la sensibilidad. 
        \begin{itemize}
            \item Primeras 100 páginas expone la parte de las formas a priori de la sensibilidad.
            \item Hay conceptos que son categorías, pero que están vinculadas a la experiencia sensible. 
            \item Cuando dice ``el tiempo y el espacio'', el tiempo solo es empírico, el tiempo surge de la experiencia sensible; lo mismo con el espacio, como condición espacial para que existan y se manifiesten objetos. Conceptos un poco lo de Leibniz de la Mónada.
            \item Esta es la idea de la sustancia de Aristóteles. 
            \item La filosofía para Kant tenía que tener un fundamento crítico, no se trata de volver a Santo Tomás y Aristóteles, es de recuperar una tradición después de un trauma (Descartes). 
            \item La metafísica tradicional necesita antes de una noción crítica. 
            \item Kant intenta mantener un equilibrio entre lo ideal y lo material.
        \end{itemize}
    \item Segunda critica (crítica de la razón práctica): el Faktum moral.
        \begin{itemize}
            \item 
        \end{itemize}
    \item Tercera critica (crítica del juicio): estética y filosofía de la naturaleza. 
        \begin{itemize}
            \item En el sentido griego del término estético.
        \end{itemize}
    \item El último Kant: la historia y la religión: \pregunta{qué esperar} en este y el otro mundo. 
\end{enumerate}

